\section{Etat des lieux des définitions sur l'intelligence artificielle}\label{appendice:definition-ia}

\begin{tcolorbox}[title= définitions anglaises du Cambridge dictionary]
\begin{enumerate}
    \item {\textbf{intelligence} \\
\textit{noun} \\
the ability to learn, understand, and make judgments or have opinions that are based on reason}
    \item {
    \textbf{artificial} \\
\textit{adjective}\\
made by people, often as a copy of something natural
    }
    \item  {
    \textbf{artificial intelligence}\\
    \textit{noun}\\
the study of how to produce machines that have some of the qualities that the human mind has, such as the ability to understand language, recognize pictures, solve problems, and learn.
    }
\end{enumerate}





\end{tcolorbox}


\begin{tcolorbox}[title= Définitions françaises du dictionnaire ATILF]
\begin{enumerate}
    \item {\textbf{INTELLIGENCE}, \textit{subst. fém.}\\
I. [Chez les êtres animés] Fonction mentale d'organisation du réel en pensées chez l'être humain, en actes chez l'être humain et l'animal.
}
    \item {\textbf{ARTIFICIEL, ELLE}, \textit{adj. et subst.} \\
I. \textit{Adjectif} \\
A. Qui est dû à l'art, qui est fabriqué, fait de toutes pièces; qui imite la nature, qui se substitue à elle; qui n'est pas naturel.
}
    \item {
\textit{LOG., INFORM.} \textbf{Intelligence artificielle}, Recherche de moyens susceptibles de doter les systèmes informatiques de capacités intellectuelles comparables à celles des êtres humains`` (La Recherche, janv. 1979, no 96, vol. 10, p. 61).
}
\end{enumerate}



\end{tcolorbox}


\begin{tcolorbox}[title= Définitions françaises des dictionnaire Littré]
\begin{enumerate}
    \item {\textbf{intelligence}
\textit{(in-tèl-li-jan-s') s. f.}
Qualité de ce qui est intelligent ; faculté de comprendre.
}
    \item {\textbf{artificiel, elle}
\textit{(ar-ti-fi-si-èl, è-l') adj.}
Qui se fait par art, opposé à naturel.
}
\end{enumerate}

\end{tcolorbox}
\pagebreak
