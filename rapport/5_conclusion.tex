\section{Conclusion}

La démocratisation de l'\gls{ia} produisant de plus en plus de résultats spectaculaires comme DALL-E2, Midjourney et Stable Diffusion, s'accélère à grand pas ces derniers mois, à tel point que se pose la question du futur de la composition artistique. Plus nombreuses sont les avancées technologiques, plus il apparaît important de prendre le temps de réfléchir à leur sens, à leur utilisation pour le service du bien commun. Ce temps de pause doit permettre d'évaluer, d'améliorer l'existant pour en faire des solutions plus éthiques. Pour cela, il est nécessaire d'abonder le champ de la recherche dans l'éthique de l'\gls{ia} et les projets mettant en oeuvre une démarche éthique.

Dans cette optique, comprendre puis enrichir le projet de l'association Tournesol constitue une premier pas vers la recherche dans l'éthique de l'\gls{ia}. En effet, ces travaux ont été l'occasion de réfléchir sur l'éthique, d'en saisir les tenants et les aboutissants. Cela passe par la compréhension de la modélisation statistique et de l'algorithme Mehestan, puis par la mise en place d'une heuristique en ligne et de son évaluation. Cette nouvelle fonctionalité s'insère dans un existant en production, avec la difficulté de se synchroniser régulièrement sur cet existant évoluant de jour en jour. 
Afin de mettre en production la nouvelle fonctionnalité, il a été nécessaire de s'assurer des garanties théoriques et d'évaluer de manière pratique les écarts et approximations d'une telle heuristique.

Pour cela, dans un premier temps, des tests déterministes donnent une première estimation de la qualité de l'heuristique, à travers le biais de somme nulle et l'écart à l'algorithme Mehestan. Or, il n'est pas possible de simuler tous les cas. Ainsi, une approche par simulation de Monte-Carlo donnent une image plus précise des écarts. Enfin, une dernière approche a été mise en place en jouant aléatoirement un tir sur données réelles et à volumétrie importante.

Les résultats sont satisfaisants par rapport aux enjeux, bien que le temps de calcul de l'heuristique ne permette pas une intégration pour tous les contributeurs. En effet, le risque de conflit entre deux heuristiques pour deux contributeurs simultanés augmente avec la volumétrie de la base de données. Or, la première version de l'heuristique ne permet pas de gérer efficacement ces conflits, avec notamment des solutions de verrou. Après audit du code, l'équipe de Tournesol songe à activer l'heuristique uniquement pour les contributeurs ayant le plus de comparaisons, afin de permettre de mettre à jour en direct les scores de Tournesol et d'afficher une actualisation en temps réel plutôt que toutes les six heures.
\pagebreak