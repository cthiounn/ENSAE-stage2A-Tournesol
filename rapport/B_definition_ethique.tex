\section{Etat des lieux des définitions sur l'éthique des \gls{ia}}

\begin{tcolorbox}[title= Définitions françaises de l'éthique du dictionnaire ATILF et Littré]
\begin{enumerate}
\item{
\textbf{ÉTHIQUE}, \textit{subst. fém. et adj.} \\
I. Subst. fém.\\
A. PHILOS. Science qui traite des principes régulateurs de l'action et de la conduite morale
}
\item{
\textbf{éthique} \\
\textit{(é-ti-k') s. f.} \\
1 Terme de philosophie. La science de la morale.
L'éthique politique a deux objets principaux : la culture de la nature intelligente, l'institution du peuple, Diderot, Opin. des anc. philos. (Sarrasins).
Les Éthiques, titre d'un ouvrage d'Aristote qui traite de la morale.
Tel est le traité des caractères de mœurs que nous a laissé Théophraste ; il l'a puisé dans les Éthiques d'Aristote, dont il fut le disciple, La Bruyère, Disc. sur Théophr.

2 Adj. Qui appartient à la morale. Préceptes éthiques.
}
\end{enumerate}
\end{tcolorbox}



\pagebreak
