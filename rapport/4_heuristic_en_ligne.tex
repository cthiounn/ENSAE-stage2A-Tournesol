\section{Heuristiques en ligne}
Afin de permettre un calcul rapide, les heuristiques sont un compromis entre rapidité et fiabilité. En effet, une heuristique est une méthode de calcul rapide, sans toutefois être exact ou optimal.

\subsection{Première heuristique}

Les conditions du premier ordre de la maximisation de la vraisemblance a posteriori donnent :


\subsection{Proposition d'estimateurs mesurant la qualité de l'heuristique}


\subsubsection{Biais de somme nulle}

Les comparaisons relatives constituent un système à somme nulle, puisque des quantités sont ajoutées et retirées de manière asymétrique. Les scores individuels bruts sont à somme nulle. En cas de somme non nulle, le biais signifie que des scores sont sous-estimés ou sur-estimés sans justification valable. Par conséquent, par exemple, si les scores sont d'une entité sont sous-estimés par plusieurs contributeurs, alors le score global de l'entité peut être lui aussi sous-estimé. Il convient alors de mesurer et de maîtriser cette déformation des poids dans un premier temps d'une manière grossière.

Soit l'indicateur \gls{bsn},
$BSN=$

\subsubsection{Somme des écarts absolus entre résultats de l'heuristique et de Mehestan}

Comme l'heuristique permet d'avoir des résultats approchés des scores calculés Mehestan, la somme des écarts absolus constitue une mesure de la convergence entre les deux algorithmes. Idéalement, cet indicateur devrait tendre vers zéro si possible. Dans le cas contraire, il est nécessaire de mesurer à quel point les résultats de l'heuristique s'écartent des scores réels et d'affiner l'heuristique pour diminuer l'écart.

Soit l'indicateur \gls{eam}
$EAM=$


De plus, la somme des écarts absolus des incertitudes donne une idée de la déformation de l'intervalle de confiance.


En outre, la somme des écarts sous-estimés et la somme des écarts sur-estimé constituent des mesures complémentaires riches en information.

\subsubsection{Temps de rapidité de l'heuristique}

Une heuristique doit être suffisamment rapide pour permettre d'être pour la plateforme Tournesol, joué plusieurs fois par contributeur et en concurrence avec d'autres contributeurs.


\subsection{Calcul théorique de l'heuristique proposée}
%\subsubsection{Evaluation des bornes sur l'estimateur \gls{BSN}}

1. Ajout d'un premier score 

Soit un contributeur \textit{i} ayant fait une comparaison entre $E_{1}$ et $E_{2}$ avec une notation $n_{1-2} \in [\![-10;10]\!]$

$r_{t+1}= \begin{pmatrix}
0 & n_{1-2} \\
-n_{1-2} & 0 
\end{pmatrix}$

$l_{t+1}= \begin{pmatrix}
0 & l_1 \\
-l_1 & 0 \\
\end{pmatrix}
$

$k_{t+1}= \begin{pmatrix}
0 & k_1 \\
k_1 & 0 \\
\end{pmatrix}
$


$L_{t+1}= \begin{pmatrix}
l_1 k_1\\
-l_1 k_1 \\
\end{pmatrix} \triangleq
\begin{pmatrix}
L1\\
L2
\end{pmatrix} 
$

$K_{aa,t+1}= \begin{pmatrix}
\alpha + k_1\\
\alpha +  k_1\\
\end{pmatrix} \triangleq
\begin{pmatrix}
K1\\
K2
\end{pmatrix} 
$

avec les scores bruts individuels calculés par l'algorithme Mehestan

$\score_{t+1}= \begin{pmatrix}
\score_{t+1,1} \\
\score_{t+1,2} 
\end{pmatrix}$

avec $\score_{t+1,1}  = -\score_{t+1,2} = l_1 k_1/(\alpha+2 k_2)$

L'heuristique donne pour $\tau=0, A^{t+1,0}=\set{E_1,E_2}$

$ \scoreh_{t+1,0}= \begin{pmatrix}
(L_1+\score_{1} k_1) / K_1 \\
(L_2+\score_{2} k_1) / K_2
\end{pmatrix}$

pour $\tau=1, A^{t+1,1}=\set{E_1,E_2}$

$ \scoreh_{t+1,1}= \begin{pmatrix}
(L_1+\scoreh_{t+1,0,1} k_1) / K_1 \\
(L_2+\scoreh_{t+1,0,2} k_1) / K_2
\end{pmatrix}$


BSN=0

car  $(L_1+\score_{1} k_1) / K_1 =
-(L_2+\score_{2} k_1) / K_2$

puis 
$(L_1+\scoreh_{t+1,0,1} k_1) / K_1 =
-(L_2+\scoreh_{t+1,0,2} k_1) / K_2 $

EAM=  $|\score_{t+1,1} - \scoreh_{t+1,1,1}|   +  |\score_{t+1,2} - \scoreh_{t+1,1,2}| $

$\forall{n_{1-2}}\in [\![-10;10]\!], EAM \in [\![0,0.842]\!]$ 
min et max en 0 et {-8,8}

%%%%%%%%%%%%%%%%%%%%%%%%%%%% Cas 3dim %%%%%%%%%%%%%%%%%%%%%%%%%%%%%%%%%%


2. Ajout d'un score après calcul par Mehestan

Soit un contributeur \textit{i} ayant fait une comparaison entre $E_{1}$ et $E_{2}$ avec une notation $n_{1-2} \in [\![-10;10]\!]$


$r_{t}= \begin{pmatrix}
0 & n_{1-2} \\
-n_{1-2} & 0 
\end{pmatrix}$

avec les scores bruts individuels calculés par l'algorithme Mehestan

$\score_{t}= \begin{pmatrix}
\score_{t,1} \\
\score_{t,2} 
\end{pmatrix}$

Le contributeur effectue une nouvelle comparaison entre $E_{2}$ et $E_{3}$ avec une notation $n_{2-3} \in [\![-10;10]\!]$. Les scores sont cette fois-ci calculés par l'heuristique.

$r_{t+1}= \begin{pmatrix}
0 & n_{1-2} & 0\\
-n_{1-2} & 0 & n_{2-3}\\
0 & -n_{2-3} & 0
\end{pmatrix}$

$l_{t+1}= \begin{pmatrix}
0 & l_1 & 0\\
-l_1 & 0 & l_2\\
0 & -l_2 & 0
\end{pmatrix}
$

$k_{t+1}= \begin{pmatrix}
0 & k_1 & 0\\
k_1 & 0 & k_2\\
0 & k_2 & 0
\end{pmatrix}
$

avec 

$l_1= -\tilde{r_1}/\sqrt{(1-\tilde{r_1}²)}$

et $k_1= (1-\tilde{r_1}²)³ $

et $\tilde{r_1} = n_{1-2}/(R_{max}+1)$

et ici $R_{max}=10$

$l_{t+1} \cdot k_{t+1}= \begin{pmatrix}
0 & l_1 k_1 & 0\\
-l_1 k_1 & 0 & l_2 k_2\\
0 & -l_2 k_2 & 0
\end{pmatrix}
$

$L_{t+1}= \begin{pmatrix}
l_1 k_1\\
-l_1 k_1 +l_2 k_2\\
-l_2 k_2
\end{pmatrix} \triangleq
\begin{pmatrix}
L1\\
L2\\
L3
\end{pmatrix} 
$

$K_{aa,t+1}= \begin{pmatrix}
\alpha + k_1\\
\alpha +  k_1 + k_2\\
\alpha + k_2
\end{pmatrix} \triangleq
\begin{pmatrix}
K1\\
K2\\
K3
\end{pmatrix} 
$



Pour $\tau=0, A^{t+1,0}=\set{E_2,E_3}$

$ \scoreh_{t+1,0}= \begin{pmatrix}
(L_2+\score_{1} k_1) / K_2 \\
(L_3+\score_{2} k_2) / K_3
\end{pmatrix}

avec $\score_{1} = - \score_{2} = l_1 k_1/(\alpha + 2 k_1)&

Pour $\tau=1, A^{t+1,1}=\set{E_1,E_2,E_3}$

$ \scoreh_{t+1,1}= \begin{pmatrix}
(L_1+\scoreh_{t+1,0,2} k_1) / K_1 \\
(L_2+\score_{1} k_1) / K_2 \\
(L_3+\scoreh_{t+1,0,2} k_2) / K_3 
\end{pmatrix}

$BSN =
(L_1+\scoreh_{t+1,0,2} k_1) / K_1 +
(L_2+\score_{1} k_1 + \scoreh_{t+1,0,2} k_2)  / K_2 +
(L_3+\scoreh_{t+1,0,2} k_2) / K_3 
$

$\forall{n_{1-2},n_{2-3}}\in [\![-10;10]\!]² , BSN \in [-1.10,1.10]$

min et max en (-4, -9) et (4, 9)


scores de Mehestan

$K_{t+1} \score_{t+1}= L_{t+1}$

$\score_{t+1}=K^{-1}_{t+1} L$

avec 
$ K_{t+1}= \begin{pmatrix}
K_1&-k_{12}&-k_{13}\\
-k_{12}&K_2&-k_{23}\\
-k_{13}&-k_{23}&K_3
\end{pmatrix}

ici, $k_{12}=k1, k_{13}=0, k_{23}=k2$
alors, 

$ K_{t+1}= \begin{pmatrix}
K_1&-k_1&0\\
-k_1&K_2&-k_2\\
0&-k_2&K_3
\end{pmatrix}

$ K^{-1}_{t+1}= \left( \begin{array}{ccc} K_1 K_2-k_2^2 & k_1 K_2 & k_1 k_2 \\ k_1 K_2 & K_2 K_3 & k_2 K_3 \\ k_1 k_2 & k_2 K_3 & K_1 K_3-k_1^2 \end{array} \right)/(-K_2 k_1^2-k_2^2 K_3+K_1 K_2 K_3)$


EAM=  $|\score_{t+1,1} - \scoreh_{t+1,1,1}|   +  |\score_{t+1,2} - \scoreh_{t+1,1,2}|  +  |\score_{t+1,3} - \scoreh_{t+1,1,3}|  $


$\forall{n_{1-2},n_{2-3}}\in [\![-10;10]\!]²  EAM \in [ 0.000,1.640 ]$ 
min et max en ( 0,0 ) et  (-9 ,-7 )



%%%%%%%%%%%%%%%%%%%%%%%%%%%%%%% VARIANTE %%%%%%%%%%%%%%%%%%%%%%%%%%%%%%%%





Variante : 
Le contributeur effectue une nouvelle comparaison entre $E_{1}$ et $E_{3}$ avec une notation $n_{1-3} \in [\![-10;10]\!]$. Les scores sont cette fois-ci calculés par l'heuristique.

$r_{t+1}= \begin{pmatrix}
0 & n_{1-2} &  n_{1-3} \\
-n_{1-2} & 0 & 0\\
 -n_{1-3}  & 0 & 0
\end{pmatrix}$

$l_{t+1}= \begin{pmatrix}
0 & l_1 & l_2\\
-l_1 & 0 & 0\\
-l_2 & 0 & 0
\end{pmatrix}
$

$k_{t+1}= \begin{pmatrix}
0 & k_1 &  k_2 \\
k_1 & 0 & 0 \\
 k_2 & 0 & 0
\end{pmatrix}
$


$l_{t+1} \cdot k_{t+1}= \begin{pmatrix}
0 & l_1 k_1 & l_2 k_2\\
-l_1 k_1 & 0 & 0\\
-l_2 k_2 & 0 & 0
\end{pmatrix}
$

$L_{t+1}= \begin{pmatrix}
l_1 k_1 +l_2 k_2 \\
-l_1 k_1\\
-l_2 k_2
\end{pmatrix} \triangleq
\begin{pmatrix}
L1\\
L2\\
L3
\end{pmatrix} 
$

$K_{aa,t+1}= \begin{pmatrix}
\alpha + k_1 + k_2\\
\alpha +  k_1 \\
\alpha + k_2
\end{pmatrix} \triangleq
\begin{pmatrix}
K1\\
K2\\
K3
\end{pmatrix} 
$



Pour $\tau=0, A^{t+1,0}=\set{E_1,E_3}$

$ \scoreh_{t+1,0}= \begin{pmatrix}
(L_1+\score_{2} k_1) / K_1 \\
(L_3+\score_{1} k_2) / K_3
\end{pmatrix}

avec $\score_{1} = - \score_{2} = l_1 k_1/(\alpha + 2 k_1)&

Pour $\tau=1, A^{t+1,1}=\set{E_1,E_2,E_3}$

$ \scoreh_{t+1,1}= \begin{pmatrix}
(L_1+\score_{2} k_1 + \scoreh_{t+1,0,3} k_2) / K_1 \\
(L_2+\scoreh_{t+1,0,1} k_1) / K_2 \\
(L_3+\scoreh_{t+1,0,1} k_2) / K_3 
\end{pmatrix}

$BSN =
(L_1+\score_{2} k_1 + \scoreh_{t+1,0,3} k_2) / K_1 +
(L_2+\scoreh_{t+1,0,1} k_1) / K_2 +
(L_3+\scoreh_{t+1,0,1} k_2) / K_3 
$

$\forall{n_{1-2},n_{1-3}}\in [\![-10;10]\!]² , BSN \in [-1.10,1.10]$


min et max en (4, -9) et (-4, 9)

scores de Mehestan

$K_{t+1} \score_{t+1}= L_{t+1}$

$$\score_{t+1}=K^{-1}_{t+1} L$

avec 
$ K_{t+1}= \begin{pmatrix}
K_1&-k_{12}&-k_{13}\\
-k_{12}&K_2&-k_{23}\\
-k_{13}&-k_{23}&K_3
\end{pmatrix}

ici, $k_{12}=k_1, k_{13}=k_2, k_{23}=0$
alors, 

$ K_{t+1}= \begin{pmatrix}
K_1&-k_1&-k_2\\
-k_1&K_2&0\\
-k_2&0&K_3
\end{pmatrix}

$ K^{-1}_{t+1}= \left( \begin{array}{ccc} K_1 K_2 & k_1 K_2 & k_2 K_1 \\ k_1 K_2 & K_2 K_3-k_2^2 & k_1 k_2 \\ k_2 K_1 & k_1 k_2 & K_1 K_3-k_1^2 \end{array} \right)/(-K_2 k_1^2-k_2^2 K_1+K_1 K_2 K_3)$

EAM=  $|\score_{t+1,1} - \scoreh_{t+1,1,1}|   +  |\score_{t+1,2} - \scoreh_{t+1,1,2}|  +  |\score_{t+1,3} - \scoreh_{t+1,1,3}|  $


$\forall{n_{1-2},n_{2-3}}\in [\![-10;10]\!]²  EAM \in [0.000,1.640 ]$ 
min et max en ( 0,0 ) et  (-9 ,7 )

\subsection{Evaluation pratique de la qualité de l'heuristique}
\subsubsection{Evaluation par des tests déterministiques}

Dans un premier temps, les tests Django mis en place adhoc permettent de tester des cas simples, mais non exhaustifs de l'ensemble de l'espace des possibilités. Ces premiers tests permettent d'avoir un premier aperçu du comportement de l'heuristique.

Ex-nihilo : ajout d'une comparaison à 10
Depuis une comparaison : ajout d'une nouvelle comparaison 10, modification de la comparaison à 0, suppression de la comparaison
Ex-nihilo : ajout séquentielle de toutes les comparaisons sur 10 entités à 10
Depuis la situation précédente : modifications séquentielles des comparaisons à -10

\subsubsection{Evaluation par des simulations de Monte-Carlo}

Afin d'avoir une meilleure idée de la qualité de l'heuristique, plutôt que de mettre en place tous les tests déterministiques avec des valeurs fixes de notation et de choix d'entités, les simulations de Monte-Carlo

Pour des raisons de temps, la démarche s'est focalisée sur l'ajout, mais pourrait s'appliquer pour de la modification et de la suppression, voire d'ajouter une loi sur l'action du contributeur (ajout, modification, suppression). 
1. Profil uniforme
2. Profil Gaussien
3. Profil extrême
\subsubsection{Evaluation sur données réelles}

1. Chargement de la base de données publiques

Réinitialisation de la base de données et chargement de la base de données publiques par Django (branche Adrien)

2. Modifications préalable

Afin de faciliter l'automatisation, les noms d'utilisateurs des contributeurs ont été renommés de user1 à usern et leur mot de passe réinitialisé à "tournesolpassword". De manière similaire, les entités sont renommés de yt:00000000001 à yt:000000nnnnn. Les contributeurs ayant plus de 1000 comparaisons ont été promus "supertrusted", attribut non publique servant de base de sondage pour la mise en place du réétalonnage.

3. Tirage de l'échantillon

lois uniformes

4. Tirs

xh

\pagebreak