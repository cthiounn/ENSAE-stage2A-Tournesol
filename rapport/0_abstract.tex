\section*{Note de synthèse}
\subsection*{Mise en place et évaluation d'une heuristique en ligne sur l'algorithme de Tournesol}

\textbf{Objectifs, données et contexte.} Dans un contexte où l'\acrfull{ia} prend une part de plus en plus importante dans une société toujours plus connectée, ce stage s'inscrit dans une démarche de réflexion éthique. En effet, l'objectif de ce stage est multiple. Il s'agit dans un premier temps de découvrir l'éthique de l'\gls{ia}, puis de comprendre la démarche de l'association Tournesol dont notamment son algorithme de recommandation, ses forces et ses faiblesses. Enfin, il s'agit de mettre en place une solution heuristique en ligne et de l'évaluer non seulement théorique mais aussi pratique. Pour cela, l'évaluation pratique de l'heuristique developée tire parti des données publiques de Tournesol. \\
\textbf{Mise en place de l'heuristique.} Après une revue de litterature sur l'éthique de l'\gls{ia} et sur les garanties théoriques de la \gls{qrmed} et des algorithmes robustes, la première étape a été de s'approprier l'algorithme Mehestan déjà existant et son implémentation en Python. La seconde étape a été de comprendre l'algorithme de descente de coordonnées afin de développer l'heuristique en Python avec la librairie Django. Il a également nécessaire d'effectuer les raccords avec l'existant et de s'assurer de son intégration fonctionnelle. \\
\textbf{Proposition d'indicateurs pour l'évaluation de l'heuristique.} Une fois l'heuristique développée en Python, l'étape suivante a été de mesurer sa qualité. À  cette fin, plusieurs indicateurs ont été mis en place. Le premier indicateur, le \gls{bsn}, mesure l'introduction éventuelle d'un biais dans un jeu à somme nulle. Le second indicateur, l'\gls{eam}, mesure l'écart entre les résultats de l'heuristique et les résultats de l'algorithme Mehestan, considérés comme résultats de référence. Enfin, le troisième indicateur, le \gls{tdr}, mesure la vitesse de calcul de l'heuristique. La vitesse est un point important car elle permet d'être jouée en ligne, \textit{i.e.} fréquemment et en concurrence avec d'autres utilisateurs. \\
\textbf{Simulations déterministes et stochastiques pour l'évaluation.} Le fonctionnement de la plateforme Tournesol, bien que simple par ses fonctionnalités, est complexe à modéliser. Il est possible de modéliser quelques cas, comme l'ajout de données en nombre fixe limité et à valeur déterminée. Cependant, ces quelques cas ne permettent pas de saisir l'entiereté de la complexité des cas possibles. Pour cela, une approche de Monte-Carlo utilisant plusieurs aléas, dites stochastiques, permet de mieux comprendre les distributions des indicateurs de qualité, en réitérant plusieurs fois un nombre important de simulations.\\
\textbf{Résultats.} L'heuristique introduit un biais de somme nulle à court terme, sur un faible nombre d'actions. Ce biais s'atténue avec un nombre d'actions plus important. De même, l'écart avec les résultats de l'algorithme Mehestan suit une tendance similaire. Quant à la rapidité de l'heuristique, l'heuristique est jouée en moins d'une seconde sur données réelles. La partie calcul des scores individuels se déroule en 4 ms, alors que la sauvegarde des scores et l'application du réétalonnage se déroulent respectivement en 100 ms et 250 ms. \\
\textbf{Discussion.} Les résultats de l'heuristique semblent converger vers les résultats de l'algorithme Mehestan, avec un grand nombre d'itérations. Néanmoins, avec un faible nombre d'itérations, le biais de somme nulle et l'écart de résultats peut introduire un biais dans les résultats finaux, malgré l'utilisation de statistiques robustes telles que \gls{qrmed}.
Intrinsèquement au nombre d'itérations, la rapidité de l'heuristique est relative à la volumétrie de la base de données. Plus il y a de données, plus l'heuristique prendra du temps pour se terminer. La formule de l'heuristique permet de réduire le nombre d'opérations par rapport à celui de l'algorithme Mehestan, mais les coûts fixes associés à la sauvegarde des scores et à l'application du réétalonnage sont un frein à l'utilisation concurrente de l'heuristique.  \\
\textbf{Conclusion et perspectives.} Les objectifs du stage, à savoir découvrir l'éthique de l'\gls{ia} et contribuer à l'association Tournesol en mettant en place une heuristique en ligne tout en l'évaluant, ont été satisfaits. L'heuristique produit des résultats acceptables en termes de rapidité et de qualité de convergence vers les résultats de l'algorithme Mehestan. L'association Tournesol songe à mettre en production la solution heuristique pour une partie des utilisateurs, à savoir les utilisateurs les plus contributifs.
\pagebreak

\section*{Abstract}
\subsection*{Implementing and evaluating an online heuristic on the Tournesol algorithm}

\textbf{Goals, data and context.} Nowadays, Artificial Intelligence (AI) is everywhere. In fact, in an increasingly connected society, it takes a more and more crucial role. Thus, this internship is a beginning step in direction of AI ethics. At such, it had many goals in this field. First, the aim was not only to
discover AI ethics, but also to understand the Tournesol association's approach, including
its recommendation system algorithm, its strengths and weaknesses. Secondly, the main goal was to set up an
online heuristic system and evaluate its quality in a theoretical and practical way. For the practical evaluation, Tournesol's public data was fully used.\\
\textbf{Setting up the heuristic.} After a literature review on AI ethics and the theoretical guarantees of the Quadratic Regularized Median (QRMed) and 
robust algorithms, the first stage was to fully understand the already existing Mehestan algorithm and its
implementation in Python. In order to implement the online heuristic, the second stage was first to understand the
coordinate descent algorithm. Then, it was to implement it in Python with Django library. Ultimately, the last stage was to ensure its integration within the existing code.\\
\textbf{Defining quality indicators for the heuristic evaluation.} Once the heuristic implemented in Python, the next step was to measure its quality. We defined three indicators to measure the quality of the heuristic. The first indicator, the Zero-Sum Bias (ZSB), is assessing the eventual bias in a zero-sum game. The second indicator, the Absolute Mehestan Deviation (AMD), is evaluating
the difference between the results of the heuristic and the results of the Mehestan algorithm, considered as baseline results. Finally, the third indicator, the Speed Time (TDR), is measuring the speed of the heuristic. The speed is an important point since faster computation guarantees the safeness of the heuristic to be played online frequently and without lock conflicts between other users.\\
\textbf{Deterministic and stochastic simulations for evaluation.} Tournesol's processes are complex to evaluate, even though the functionalities are simple. It is possible to evaluate a few cases, with a limited number of fixed-value data and additions. Unfortunately, it is not sufficient to grasp the whole complexity of Tournesol's processes. However, with a random-based approach called stochastic, Monte-Carlo simulations allow us to understand better the distributions of the quality 
indicators, by repeating several times a large number of simulations.\\
\textbf{Results.} The heuristic introduces a zero-sum bias in the short run, on a small number of operations. This bias diminishes with a larger number of operations. Similarly, the AMD indicator follows the same trend. Concerning the speed part of the heuristic, the heuristic is played
in less than a second on real data. The individual scores' computation completes in  4
ms, while saving the scores in the database and the recalibration computation finish in 100 ms and 250 ms respectively. \\
\textbf{Discussion.} The results of the heuristic seem to converge with the results of the Mehestan algorithm, with a large number of iterations. Nevertheless, with a low number of iterations, both the zero-sum bias and the AMD bias can result in a
bias in the final results, despite the use of robust statistics such as QRMed. Independently of the number of iterations, the speed of the heuristic
is proportional to the volume of the database. The more data there is, the longer the heuristic will take to compute. The heuristic formula allows reducing the number of operations compared to the Mehestan algorithm. However, the fixed costs associated with saving the scores and the recalibration are a hindrance to the full use of the heuristic in production without lock conflicts between users.\\
\textbf{Conclusion and possible outcomes.} The objectives of the internship - namely to discover the ethics of AI and to contribute to the Tournesol association by implementing an online heuristic with a pratical and theoretical evaluation - have been met.
The heuristic produces sufficient results both in terms of speed and quality of convergence to the results of the Mehestan algorithm. The Tournesol association is currently thinking about delivering the heuristic solution into production regarding the most contributing users only.
\pagebreak