\section{Tournesol et Mehestan : construction d'un algorithme robuste sur des données collaboratives dans un monde éthique}

\subsection{Présentation}

La plateforme collaborative Tournesol, accessible en ligne avec un navigateur à l'adresse suivante \href{https://tournesol.app}{https://tournesol.app}, est une plateforme en ligne, où chacun peut contribuer en effectuant des comparaisons entre deux entités (vidéos, candidats). Pour cela, l'internaute doit au préalable s'inscrire sur la plaforme puis s'authentifier. Une fois authentifié, en tant que contributeur, il a la possibilité de comparer deux entités, de gérer une liste d'entités à comparer, de visualiser ses comparaisons. Tout internaute peut visualiser les recommandations de la plateforme, après calcul et classement par l'algorithme Mehestan, recommandations générales ou par critère. Aujourd'hui, la plateforme Tournesol est principalement centrée sur la comparaison et la recommandation de vidéos hébergées sur Youtube, une source d'informations aussi imparfaite soit-elle, mais peut très bien proposer des comparaisons sur d'autres champs, tel que le choix d'un candidat comme pour l'élection présidentielle 2022, ou pour des revues par les pairs d'articles scientifiques.
A partir de l'ensemble des comparaisons deux à deux d'entités par des utilisateurs avec des notations différentes, l'algorithme Mehestan a pour d'harmoniser toutes les notations deux à deux et de calculer un score global pour chaque entité de manière robuste. En effet, un contributeur ne devrait pas être un contributeur pivot\footnote{en microéconomie, un agent est dit pivot si sa présence ou son absence caractèrise le résultat de production d'un bien public } d'un score d'une entité, afin d'éviter des stratégies sous-optimales et des attaques mal intentionnées


\subsection{Modélisation statistique}
\pagebreak

