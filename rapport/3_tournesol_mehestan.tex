\section{Tournesol et Mehestan : construction d'un algorithme robuste sur des données collaboratives dans un monde éthique}

\subsection{Présentation générale}

La plateforme collaborative Tournesol, accessible en ligne avec un navigateur à l'adresse suivante \href{https://tournesol.app}{https://tournesol.app}, est une plateforme en ligne, où chacun peut contribuer en effectuant des comparaisons entre deux entités (vidéos, candidats). Pour cela, l'internaute doit au préalable s'inscrire sur la plateforme puis s'authentifier. Une fois authentifié, en tant que contributeur, il a la possibilité de comparer deux entités, de gérer une liste d'entités à comparer, de visualiser ses comparaisons. Tout internaute peut visualiser les recommandations de la plateforme, après calcul et classement par l'algorithme Mehestan, recommandations générales ou par critère.

Aujourd'hui, la plateforme Tournesol est principalement centrée sur la comparaison et la recommandation de vidéos hébergées sur Youtube, une source d'informations aussi imparfaite soit-elle, mais peut très bien proposer des comparaisons sur d'autres champs, tel que le choix d'un candidat comme pour l'élection présidentielle 2022, ou pour des revues par les pairs d'articles scientifiques.

À partir de l'ensemble des comparaisons deux par deux d'entités par des utilisateurs avec des notations différentes, l'algorithme Mehestan a pour objectif d'harmoniser toutes les notations deux à deux et de calculer un score global pour chaque entité de manière robuste. En effet, un contributeur ne devrait pas être un contributeur pivot\footnote{en microéconomie, un agent est dit pivot si sa présence ou son absence caractèrise le résultat de production d'un bien public.} d'un score d'une entité, afin d'éviter des stratégies sous-optimales et des attaques mal intentionnées de manipulation de l'information.

\subsection{Objectif de Tournesol : "A better attention is all we need"}

Les scores calculées permettent dans un premier temps de proposer une page de recommandation générale. L'utilisateur a la possibilité de filtrer par critères ou de donner des poids différents aux critères afin de construire sa propre page de recommandation. Dans un monde où la volumétrie de l'information est difficilement appréhendable pour le cerveau humain, les systèmes de recommandations actuels ne sont pas satisfaisants au regard de l'intérêt commun, car ces systèmes sont conçus pour répondre à l'intérêt privé des entreprises, \textit{i.e.} maximiser la captivité de l'attention des utilisateurs. 

L'objectif premier de Tournesol est donc de proposer des informations recommandées pour le bien commun, que la plateforme met en avant. Loin de vouloir corriger tous les biais, la plateforme Tournesol introduit un biais particulier, le biais du jugement des contributeurs, afin de réinternaliser le coût social et de mieux répondre aux préférences des utilisateurs en termes d'accès à l'information. En ce sens, l'objectif de Tournesol n'est donc pas de proposer des recommandations de manière "neutre", mais avec une visée éthique.

Les informations recommandées prennent tout d'abord la forme de recommandation globale. Le coeur de l'algorithme Mehestan est en premier lieu de proposer une quantification de la désirabilité globale des entités, par l'aggrégation robuste des préférences individuelles. L'algorithme permet alors d'avoir une relation d'ordre et de proposer cette hiérarchisation à l'ensemble des internautes \footnote{https://tournesol.app/recommendations?date=Month&language=}.

De plus, l'internaute peut personnaliser ses recommandations en choississant des filtres et en sélectionnant des critères. Tournesol propose de constituer un indicateur composite personnalisé afin d'affiner ses recommandations et ainsi de proposer des entités proches de ses préférences individuelles. Ce système de recommandations favorise alors une découvrabilité personnalisée d'entités recommandables.

Compte tenu de l'importance de la qualité de l'information, l'algorithme de Tournesol doit être robuste et sécurisé. En particulier, l'algorithme doit être robuste à la contamination de données, aux attaques coordonnées de manipulation de l'information ou attaques bizantines. De même, la plateforme doit être sécurisé et garantir l'intégrité du processus et des données soumises par les contributeur, ainsi que respecter l'éventuel anonymat choisi des préférences individuelles.


Au delà de proposer des recommandations, Tournesol a pour ambition de constituer une base de données éthique et collaborative de grande envergure, afin de devenir un standard de facto et de montrer qu'il est possible d'allier éthique et IA et de guider les entités privés à emboîter le pas. Pour cela, un des objectifs finaux de Tournesol est de promouvoir des données collaboratives éthiques en open data et spécifiquement à destination des chercheurs en science des données.

\subsection{Modélisation statistique}

Dans le cadre d'une modélisation statistique bayésienne, sous hypothèse qu'il existe pour chaque entité un score réel avec une relation d'ordre, soit $\theta_{a}\in\Theta=\mathbf{R}$ le paramètre à estimer représentant le score de l'entité a.

Le contributeur n compare deux entités a et b en notant selon un critère principal et un ou plusieurs critères secondaires optionnelles sur une échelle de $R_{min}=-10$ à $R_{max}=10$

Une comparaison est donc un 4-uplet (n, a, b, liste de couple score-critère (s,c))

Le score utilisé pour la recommandation est actuellement calculé sur le critère principal et les scores sur les autres critères secondaires sont calculées de manière approximativement similaire. Ainsi, la suite de ce rapport se focalisera sur le calcul du score du critère principal.

\subsubsection{Problème inverse bayésien}
Hypothèse : les recommandations formulées par un contributeur sont à un bruit près les véritables préférences du contributeur


formule

MAP

\subsection{Fonctionnement de l'algorithme Mehestan}

L'algorithme Mehestan fonctionne en trois phases :

\begin{enumerate}
    \item Construction des scores individuels
    \item Construction du réétalonnage et des scores normalisés
    \item Construction des scores des entités
\end{enumerate}
\subsubsection{Construction des scores individuels}

Pour chaque contributeur, à partir de sa matrice de comparaison r_t, le vecteur des scores individuels bruts et des incertitudes est calculé à partir des formules énoncées précédemment

\subsubsection{Construction du réétalonnage et des scores normalisés}

Avant d'agréger les scores individuels, il est nécessaire de réétalonner ces scores sur une même échelle.

\subsubsection{Construction des scores des entités}

\pagebreak

