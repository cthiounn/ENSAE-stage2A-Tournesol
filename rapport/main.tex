\documentclass[hidelinks, 12pt]{article}
%%%%%%%%%%%%%%%%%%%%%%%%%
%% Packages généraux
%%%%%%%%%%%%%%%%%%%%%%%%%

\usepackage[utf8]{inputenc}
\usepackage[toc,page]{appendix} % appendices
\usepackage{hyperref} % liens cliquables
\usepackage{xcolor} % couleur du texte
\usepackage{blindtext} % sections non numérotées
\usepackage[autostyle]{csquotes} % citation en ligne
\usepackage{footnotehyper}
\usepackage{comment} % pour masquer des passages
\usepackage{titling}
\usepackage{titlesec}
\usepackage{amsmath}
\usepackage[most]{tcolorbox}
\renewcommand{\appendixpagename}{Annexes}
\renewcommand{\appendixtocname}{Annexes}

\newtheorem{hypothesis}{Hypothèse}
\newtheorem{exemple}{Exemple}[section]
\newcommand{\score}{\theta^*}
\newcommand{\deltascore}{\Delta\theta^*}
\newcommand{\scoreh}{\theta^h}
\newcommand{\deltascoreh}{\Delta\theta^h}
\newcommand{\set}[1]{\left\lbrace #1 \right\rbrace}
%\newcommand{\bm}[1]{{\color{blue}\textbf{BM}: #1}}
\def\siecle#1{\textsc{\romannumeral #1}\textsuperscript{e}~siècle}

% Manager de couleurs pour les liens
\hypersetup{
  citebordercolor = {1 1 1} % White box arround citation (that we can not see
  %colorlinks   = true, %Colours links instead of  boxes
  %urlcolor     = green, %Colour for external hyperlinks
  %linkcolor    =green, %Colour of internal links
  %citecolor   = grey %Colour of citations
}

%%%%%%%%%%%%%%%%%%%%%%%%%
%% Bibliographie
%%%%%%%%%%%%%%%%%%%%%%%%%

\usepackage[
    backend=biber,
    style=apa,
    natbib
  ]{biblatex}
  
\addbibresource{biblio.bib} % import biblio file
\defbibfilter{papers}{
  type=article or
  type=techreport
}

\defbibfilter{onlinepapers}{
  type=online or
  type=InCollection
}
%%%%%%%%%%%%%%%%%%%%%%%%%
%% Glossaire
%%%%%%%%%%%%%%%%%%%%%%%%%

\usepackage[toc]{glossaries}

\makeglossaries


\newglossaryentry{fairness}
{
  name={Fairness},
  description={Caractérisation de l'équité},
}

\newglossaryentry{diffpriv}
{
  name={Differential privacy},
  description={Confidentialité différentielle},
}

\newacronym{ia}{IA}{Intelligence Artificielle}
\newacronym{gofai}{GOFAI}{Good Old Fashion Artificial Intelligence - Intelligence artificielle traditionnelle}
\newacronym{ml}{ML}{Machine Learning - Méthodes d'apprentissage}
\newacronym{dl}{DL}{Deep Learning - Méthodes d'apprentissage profond}
\newacronym{qrmed}{QRMed}{Médiane à régularisation quadratique}
\newacronym{brmean}{BRMean}{Moyenne Byzantine-robuste}
\newacronym{bsn}{BSN}{Biais de somme nulle}
\newacronym{eam}{EAM}{Écart absolu à Mehestan}
\newacronym{tdr}{TDR}{Temps de rapidité}


%%%%%%%%%%%%%%%%%%%%%%%%%
%% Gestion des marges et mise en page générale
%%%%%%%%%%%%%%%%%%%%%%%%%
\usepackage[margin=0.8in]{geometry} % Marges
\usepackage{setspace} % Espace entre les lignes 
\onehalfspacing % Interligne 1.5
% Police utilisée
\usepackage{fontspec}
\setmainfont{Times New Roman}
\setcounter{secnumdepth}{4}
\titleformat{\paragraph}{\normalfont\normalsize\bfseries}{\theparagraph}{1em}{}
\titlespacing*{\paragraph}{0pt}{3.25ex plus 1ex minus .2ex}{1.5ex plus .2ex}
%%%%%%%%%%%%%%%%%%%%%%%%%
%% Notations, code, mathématiques
%%%%%%%%%%%%%%%%%%%%%%%%%

% \usepackage{listings} % code
\usepackage{listingsutf8}
\usepackage{minted} % code coloré
\usepackage{pythontex}
\usepackage{amsmath} % symboles math
\usepackage{amssymb}
\usepackage{bbm} % dummy sign (fat one)
\usepackage{mathtools} % for equation alginment with \mathrlap
\usepackage{cancel} % for cancelation
\usepackage[scr=boondoxo]{mathalfa} % pour mathscr
\usepackage{aligned-overset} % for alignment even with overset
\usepackage{paracol}
\usepackage[lined,boxed,commentsnumbered,ruled,vlined]{algorithm2e}

\SetKwInput{KwRes}{R\'esultat}%
\SetKwIF{Si}{SinonSi}{Sinon}{si}{alors}{sinon si}{sinon}{fin si}%
\SetKwFor{Tq}{tant que}{faire}{fin tq}%
\SetKwFor{PourCh}{pour chaque}{faire}{fin pour chaque}%
%%%%%%%%%%%%%%%%%%%%%%%%%
%% Images
%%%%%%%%%%%%%%%%%%%%%%%%%

% \usepackage{subfigure}
\usepackage{graphicx}
\usepackage{float}
\usepackage{subcaption}
\graphicspath{{img/}}

%\title{Optimisation des M-estimateurs robustes de Tournesol par des heuristiques en ligne}
\title{Evaluation et mise en place d'une heuristique en ligne sur l'algorithme de Tournesol }
\author{
Conrad Thiounn }
\date{Septembre 12022}

\begin{document}

%\begin{titlingpage}
%\maketitle
%\begin{center}
%Mémoire de stage de 2A
%
%ENSAE 12022: Stage d'application
%
%\vspace{5mm}
%
%Projet encadré par Lê-Nguyên Hoang (Association Tournesol)
%\end{center}
%\end{titlingpage}

\begin{titlingpage}
\noindent
\textbf{\Large{THIOUNN Conrad}} \hfill
\textbf{\Large{ENSAE 2\textsuperscript{ème} année}} 
\begin{flushright}
\begin{minipage}{5.4cm}
\begin{center}
\textbf{\emph{\large{Stage d'application \\
Année scolaire 2021-2022 }}} 
\end{center}
\end{minipage}
\end{flushright}

\begin{flushright}
\includegraphics[scale=0.3]{ensae.png}
\end{flushright}
\vspace{5.5cm}

\hspace*{-\parindent}%
\fbox{
\begin{minipage}{17cm}
\begin{center}
\textbf{\huge{Mise en place et évaluation d'une heuristique en ligne sur l'algorithme de Tournesol}} \\
\end{center}
\end{minipage}
}

\vspace{5.5cm}

\noindent \includegraphics[scale=0.7]{logo_tournesol.png}

\vspace{0.4cm}

\noindent \textbf{\Large{Association Tournesol \hfill Maître de stage : Lê-Nguyên HOANG \\ 
\hspace{20mm} Lausanne (Suisse)}} \hfill \textbf{\large{1 juin 2022 - 30 septembre 2022}}

\end{titlingpage}


\pagenumbering{roman}

\renewcommand*\contentsname{Table des matières}
\tableofcontents

\setlength\parskip{0.2 em} % some space between paragraphs

\pagebreak
\phantomsection % Mandatory line to avoid issues with the hyperlinks in the table of contents
\addcontentsline{toc}{section}{Remerciements}
\section*{Remerciements} 

En premier lieu, j'aimerais remercier Oscar Villemaud, Lê-Nguyên Hoang, l'ENSAE Paris et l'association Tournesol de m'avoir donné l'opportunité de contribuer et d'améliorer la plateforme tournesol.app et de pouvoir contribuer à l'éthique de l'\gls{ia}.


Ensuite, je souhaiterais également remercier l'équipe de Tournesol, en particulier Louis Faucon, Adrien Matissart, Aidan Jungo et Romain Beylerian, pour leur accueil, leurs conseils, leur bienveillance et d'avoir partagé leurs connaissances en matière d'éthique de l'\gls{ia}.


En outre, je voudrais exprimer toute ma gratitude envers Ines Hiverlet et Fanny Salvan, futures administratrices de l'Insee, d'avoir consacré leur précieux temps pour une relecture attentive de mon rapport de stage et d'avoir proposer des améliorations. Leurs aides précieuses ont grandement façonné la forme et le fond de ce rapport.


Enfin, j'aimerais remercier mes proches et ma famille, qui m'ont soutenu pendant le stage et la rédaction de ce présent rapport. Car il y a toujours une probabilité non nulle qu'un oubli survienne, par défaut de mémoire, je souhaiterais presque surement remercier toutes les personnes qui ont contribué directement ou indirectement à ce rapport et qui n'ont pas été citées précédemment.



\pagebreak

\printglossary[title=Glossaire et abréviations]

\clearpage

\pagebreak
\phantomsection
\addcontentsline{toc}{section}{Notations mathématiques}
\section*{Notations}

$$
\begin{aligned}
n & \text{: l'individu n} \\
A & \text{: l'ensemble des entités notés} \\
AB & \text{: l'ensemble des couples d'entités notés} \\
N(A) & \text{: l'ensemble des autres entités liés à A par notation} \\
\end{aligned}
$$

\pagebreak
\section*{Note de synthèse}
\subsection*{Mise en place et évaluation d'une heuristique en ligne sur l'algorithme de Tournesol}

\textbf{Objectifs, données et contexte.} Dans un contexte où l'\acrfull{ia} prend une part de plus en plus importante dans une société toujours plus connectée, ce stage s'inscrit dans une démarche de réflexion éthique. En effet, l'objectif de ce stage est multiple. Il s'agit dans un premier temps de découvrir l'éthique de l'\gls{ia}, puis de comprendre la démarche de l'association Tournesol dont notamment son algorithme de recommandation, ses forces et ses faiblesses. Enfin, il s'agit de mettre en place une solution heuristique en ligne et de l'évaluer non seulement théorique mais aussi pratique. Pour cela, l'évaluation pratique de l'heuristique developée tire parti des données publiques de Tournesol. \\
\textbf{Mise en place de l'heuristique.} Après une revue de litterature sur l'éthique de l'\gls{ia} et sur les garanties théoriques de la \gls{qrmed} et des algorithmes robustes, la première étape a été de s'approprier l'algorithme Mehestan déjà existant et son implémentation en Python. La seconde étape a été de comprendre l'algorithme de descente de coordonnées afin de développer l'heuristique en Python avec la librairie Django. Il a également nécessaire d'effectuer les raccords avec l'existant et de s'assurer de son intégration fonctionnelle. \\
\textbf{Proposition d'indicateurs pour l'évaluation de l'heuristique.} Une fois l'heuristique développée en Python, l'étape suivante a été de mesurer sa qualité. À  cette fin, plusieurs indicateurs ont été mis en place. Le premier indicateur, le \gls{bsn}, mesure l'introduction éventuelle d'un biais dans un jeu à somme nulle. Le second indicateur, l'\gls{eam}, mesure l'écart entre les résultats de l'heuristique et les résultats de l'algorithme Mehestan, considérés comme résultats de référence. Enfin, le troisième indicateur, le \gls{tdr}, mesure la vitesse de calcul de l'heuristique. La vitesse est un point important car elle permet d'être jouée en ligne, \textit{i.e.} fréquemment et en concurrence avec d'autres utilisateurs. \\
\textbf{Simulations déterministes et stochastiques pour l'évaluation.} Le fonctionnement de la plateforme Tournesol, bien que simple par ses fonctionnalités, est complexe à modéliser. Il est possible de modéliser quelques cas, comme l'ajout de données en nombre fixe limité et à valeur déterminée. Cependant, ces quelques cas ne permettent pas de saisir l'entiereté de la complexité des cas possibles. Pour cela, une approche de Monte-Carlo utilisant plusieurs aléas, dites stochastiques, permet de mieux comprendre les distributions des indicateurs de qualité, en réitérant plusieurs fois un nombre important de simulations.\\
\textbf{Résultats.} L'heuristique introduit un biais de somme nulle à court terme, sur un faible nombre d'actions. Ce biais s'atténue avec un nombre d'actions plus important. De même, l'écart avec les résultats de l'algorithme Mehestan suit une tendance similaire. Quant à la rapidité de l'heuristique, l'heuristique est jouée en moins d'une seconde sur données réelles. La partie calcul des scores individuels se déroule en 4 ms, alors que la sauvegarde des scores et l'application du réétalonnage se déroulent respectivement en 100 ms et 250 ms. \\
\textbf{Discussion.} Les résultats de l'heuristique semblent converger vers les résultats de l'algorithme Mehestan, avec un grand nombre d'itérations. Néanmoins, avec un faible nombre d'itérations, le biais de somme nulle et l'écart de résultats peut introduire un biais dans les résultats finaux, malgré l'utilisation de statistiques robustes telles que \gls{qrmed}.
Intrinsèquement au nombre d'itérations, la rapidité de l'heuristique est relative à la volumétrie de la base de données. Plus il y a de données, plus l'heuristique prendra du temps pour se terminer. La formule de l'heuristique permet de réduire le nombre d'opérations par rapport à celui de l'algorithme Mehestan, mais les coûts fixes associés à la sauvegarde des scores et à l'application du réétalonnage sont un frein à l'utilisation concurrente de l'heuristique.  \\
\textbf{Conclusion et perspectives.} Les objectifs du stage, à savoir découvrir l'éthique de l'\gls{ia} et contribuer à l'association Tournesol en mettant en place une heuristique en ligne tout en l'évaluant, ont été satisfaits. L'heuristique produit des résultats acceptables en termes de rapidité et de qualité de convergence vers les résultats de l'algorithme Mehestan. L'association Tournesol songe à mettre en production la solution heuristique pour une partie des utilisateurs, à savoir les utilisateurs les plus contributifs.
\pagebreak

\section*{Abstract}
\subsection*{Implementing and evaluating an online heuristic on the Tournesol algorithm}

\textbf{Goals, data and context.} Nowadays, Artificial Intelligence (AI) is everywhere. In fact, in an increasingly connected society, it takes a more and more crucial role. Thus, this internship is a beginning step in direction of AI ethics. At such, it had many goals in this field. First, the aim was not only to
discover AI ethics, but also to understand the Tournesol association's approach, including
its recommendation system algorithm, its strengths and weaknesses. Secondly, the main goal was to set up an
online heuristic system and evaluate its quality in a theoretical and practical way. For the practical evaluation, Tournesol's public data was fully used.\\
\textbf{Setting up the heuristic.} After a literature review on AI ethics and the theoretical guarantees of the Quadratic Regularized Median (QRMed) and 
robust algorithms, the first stage was to fully understand the already existing Mehestan algorithm and its
implementation in Python. In order to implement the online heuristic, the second stage was first to understand the
coordinate descent algorithm. Then, it was to implement it in Python with Django library. Ultimately, the last stage was to ensure its integration within the existing code.\\
\textbf{Defining quality indicators for the heuristic evaluation.} Once the heuristic implemented in Python, the next step was to measure its quality. We defined three indicators to measure the quality of the heuristic. The first indicator, the Zero-Sum Bias (ZSB), is assessing the eventual bias in a zero-sum game. The second indicator, the Absolute Mehestan Deviation (AMD), is evaluating
the difference between the results of the heuristic and the results of the Mehestan algorithm, considered as baseline results. Finally, the third indicator, the Speed Time (TDR), is measuring the speed of the heuristic. The speed is an important point since faster computation guarantees the safeness of the heuristic to be played online frequently and without lock conflicts between other users.\\
\textbf{Deterministic and stochastic simulations for evaluation.} Tournesol's processes are complex to evaluate, even though the functionalities are simple. It is possible to evaluate a few cases, with a limited number of fixed-value data and additions. Unfortunately, it is not sufficient to grasp the whole complexity of Tournesol's processes. However, with a random-based approach called stochastic, Monte-Carlo simulations allow us to understand better the distributions of the quality 
indicators, by repeating several times a large number of simulations.\\
\textbf{Results.} The heuristic introduces a zero-sum bias in the short run, on a small number of operations. This bias diminishes with a larger number of operations. Similarly, the AMD indicator follows the same trend. Concerning the speed part of the heuristic, the heuristic is played
in less than a second on real data. The individual scores' computation completes in  4
ms, while saving the scores in the database and the recalibration computation finish in 100 ms and 250 ms respectively. \\
\textbf{Discussion.} The results of the heuristic seem to converge with the results of the Mehestan algorithm, with a large number of iterations. Nevertheless, with a low number of iterations, both the zero-sum bias and the AMD bias can result in a
bias in the final results, despite the use of robust statistics such as QRMed. Independently of the number of iterations, the speed of the heuristic
is proportional to the volume of the database. The more data there is, the longer the heuristic will take to compute. The heuristic formula allows reducing the number of operations compared to the Mehestan algorithm. However, the fixed costs associated with saving the scores and the recalibration are a hindrance to the full use of the heuristic in production without lock conflicts between users.\\
\textbf{Conclusion and possible outcomes.} The objectives of the internship - namely to discover the ethics of AI and to contribute to the Tournesol association by implementing an online heuristic with a pratical and theoretical evaluation - have been met.
The heuristic produces sufficient results both in terms of speed and quality of convergence to the results of the Mehestan algorithm. The Tournesol association is currently thinking about delivering the heuristic solution into production regarding the most contributing users only.
\pagebreak
\pagenumbering{arabic}
\section{Introduction}

Inondée par un flux continu d'informations diverses et potentiellement contradictoires, qu'elles proviennent d'un clair ruisseau, d'un fleuve pollué ou d'un torrent diluvien, l'humanité doit aujourd'hui résoudre l'un des plus grands défis collectifs qui lui a été posé, à savoir l'émergence d'un ensemble de crises écologiques, sanitaires et sociales, toutes aussi interconnectées que dangereuses. Malédiction ou paire d'atout, l'information découverte par la science et les sociétés à travers les siècles, par les scientifiques et les citoyens, doit permettre grâce à ce bien commun de faire émerger la meilleure solution pour répondre efficacement à ces crises. 

Encore plus aujourd'hui, l'information utilisée est source de toutes nos décisions. L'information, comme preuve intangible ou comme élément rationnel de croyance, peut aider les citoyens à mieux forger leur conception du monde et de la vérité avec une logique bayésienne. Elle est donc centrale dans le quotidien des citoyens, dans la prise de décision manuelle et individuelle, ainsi que dans la prise de décision collective et automatisée. L'automatisation de ces décisions collectives a, par corollaire, un impact important sur ses citoyens sujets et doit tendre à être le plus "juste" possible.

Néanmoins, la "justesse" des décisions prises par un algorithme s'apprécie à l'aune de la définition que chacun pourrait lui donner et peut donc se caractériser de diverses manières, notamment d'un point de vue utilitariste ou d'un point de vue éthique. D'un côté, l'utilitarisme permet de maximiser l'utilité agrégée sans toutefois prendre en compte l'équité. De l'autre côté, l'approche "éthique", \textit{a contrario}, considère plusieurs critères moraux et va au delà de l'égalité de droit en proposant une réflexion normative en avance sur le droit. De manière pratique, une décision éthique, comme pour l'éthique professionnelle, tend à rester fidèle à plusieurs valeurs (fidélité à ses sujets, fiabilité, transparence). En ce sens, dans un système démocratique, une décision éthique doit rendre compte du vote démocratique si exprimé, sinon du respect des valeurs démocratiques (liberté, égalité, fraternité, équité, solidarité).

Or, aujourd'hui, tous les algorithmes apprenants et de manière plus générale, l'\gls{ia}, à notre modeste connaissance, sont entraînés avec des données comportant des biais statistiques (de sélection, de mesure) ou non statistiques (racisme, sexisme, discrimination). Ceci amène les modèles à reproduire ces biais et à formuler des recommandations et des décisions non adaptées, aux conséquences potentiellement dangereuses. De surcroît, les données sont collectées de manière non contrôlée (scrapping en ligne de textes de communautés particulières, données dangereuses non vérifiées) et peuvent faire l'objet de compromissions par diverses entités, dans le but de manipuler un résultat ou de se dégager d'une quelconque responsabilité.

Afin d'améliorer l'éthique des \gls{ia}, l'association Tournesol a construit une plateforme collaborative \href{https://tournesol.app}{https://tournesol.app}, où les internautes sont invités à comparer selon différents critères plusieurs choix entre eux (vidéos, candidats), afin de mettre en avant l'entité qui leur semble la plus recommandable. Les données de cette plateforme permettent de mettre en place un système de recommandation robuste fondé sur un score calculé de manière transparente, robuste et sécurisé.

Cependant, les comparaisons parmi un nombre important de choix n'engendrent que des données parcimonieuses. Dès lors, l'enjeu est de reconstruire au niveau agrégé et de manière robuste tous les scores relatifs puis d'élaborer des scores absolus pour pouvoir mettre en avant l'information jugée primordiale. En somme, l'enjeu est de pouvoir quantifier la désirabilité relatif des entités à partir d'une poignée de préférences révélées de manière robuste et sécurisé.

Pour cela, l'algorithme Mehestan sur les données de Tournesol tire parti d'estimateurs basés autour de la \gls{qrmed}. Néanmoins, ces calculs ne sont pas assez rapides pour proposer des résultats au contributeur en temps réel. Afin de préserver l'engagement du contributeur, il est souhaitable d'afficher en temps réel sa participation et sa contribution au score calculé en temps réel à l'aide d'heuristiques en ligne tout en garantissant la sécurité et la fiabilité de ces calculs.

\pagebreak
\section{Etat des lieux de l'\gls{ia} et de l'éthique}

Cette section introduit les définitions qui seront utilisées dans ce rapport et qui sont indispensables à la compréhension du sujet. Comme en sociologie, les définitions sont un préalable pour une compréhension commune et éclairée et permettent la délimitation du périmètre des sujets. Au delà de l'aspect mathématique et statistique des algorithmes présentés dans ce rapport, il est tout aussi important de se questionner sur l'éthique des algorithmes et modèles déjà existants. \textit{A priori}, les termes utilisés seront, sauf exception, en français, bien que les équivalents anglophones soient davantage connus et usités. Le glossaire, en début de rapport, liste les équivalences.

\subsection{L'\gls{ia} : un, deux, trois, soleil ?}

Aujourd'hui, l'\gls{ia} est un terme célèbre, bien connu des citoyens sans toutefois avoir une idée claire des réalités que le terme désigne. Né dans la première moitié du \siecle{20}, avec l'essor de la cybernétique, l'\gls{ia} désigne let par extension sa discipline scientifique

\subsection{L'éthique : des graines à semer}

Alors que la notion d'\gls{ia} posait des problèmes en termes de représentation unifiée par chacun, l'éthique est une notion encore plus difficile à définir. En effet, l'éthique renvoie à la fois à plusieurs réalités, et également
\cite{ethique-cnil}
\pagebreak
\section{Tournesol et Mehestan : construction d'un algorithme robuste sur des données collaboratives dans un monde éthique}

\subsection{Présentation générale}

La plateforme collaborative Tournesol, accessible en ligne avec un navigateur à l'adresse
\href{https://tournesol.app}{tournesol.app}, est un espace coopératif où chacun peut contribuer en effectuant des comparaisons entre deux entités (vidéos, candidats). Après inscription et une fois authentifié, la plateforme offre la possibilité de comparer deux entités, de gérer une liste d'entités à comparer, de visualiser ses comparaisons. Tout internaute peut visualiser les recommandations générales ou par critère de la plateforme, après calcul et classement par l'algorithme Mehestan.

Aujourd'hui, la plateforme Tournesol est principalement centrée sur la comparaison et la recommandation de vidéos hébergées sur Youtube, et est capable de proposer des comparaisons sur d'autres champs, tel que le choix d'un candidat comme pour l'élection présidentielle 2022, ou pour des revues par les pairs d'articles scientifiques.


À l'aide des contributions, l'algorithme Mehestan a pour objectif d'harmoniser toutes les notations deux à deux et de calculer un score global pour chaque entité de manière robuste. En effet, un contributeur ne devrait pas être un contributeur pivot\footnote{en microéconomie, un agent est dit pivot si sa présence ou son absence caractèrise le résultat de production d'un bien public.} d'un score d'une entité, afin d'éviter des stratégies sous-optimales et des attaques mal intentionnées de manipulation de l'information.

\begin{figure}[ht]
  \includegraphics[width=\linewidth]{tournesol.png}
  \caption{Interface de la plateforme Tournesol}
  % \vspace{-15pt}
\end{figure}


\subsection{Objectif de Tournesol : "A better attention is all we need"}



Les scores calculées permettent dans un premier temps de proposer une page de recommandation générale. L'utilisateur a la possibilité de filtrer par critères ou de donner des poids différents aux critères afin de construire sa propre page de recommandation. Dans un monde où la volumétrie de l'information est difficilement appréhendable pour le cerveau humain, les systèmes de recommandations actuels ne sont pas satisfaisants au regard de l'intérêt commun, car ces systèmes sont conçus pour répondre à l'intérêt privé des entreprises, \textit{i.e.} maximiser la captivité de l'attention des utilisateurs. 

L'objectif premier de Tournesol est donc de proposer des informations recommandées pour le bien commun, que la plateforme met en avant. Loin de vouloir corriger tous les biais, la plateforme Tournesol introduit un biais particulier, le biais du jugement des contributeurs, afin de réinternaliser le coût social et de mieux répondre aux préférences des utilisateurs en termes d'accès à l'information. En ce sens, l'objectif de Tournesol n'est donc pas de proposer des recommandations de manière "neutre", mais avec une visée éthique.

Les informations recommandées prennent tout d'abord la forme de recommandation globale. Le coeur de l'algorithme Mehestan est en premier lieu de proposer une quantification de la désirabilité globale des entités, par l'aggrégation robuste des préférences individuelles. L'algorithme permet alors d'avoir une relation d'ordre et de proposer cette hiérarchisation à l'ensemble des internautes \footnote{\href{https://tournesol.app/recommendations?date=Month&language=}{https://tournesol.app/recommendations?date=Month&language=}}.

De plus, l'internaute peut personnaliser ses recommandations en choississant des filtres et en sélectionnant des critères. Tournesol propose de constituer un indicateur composite personnalisé afin d'affiner ses recommandations et ainsi de proposer des entités proches de ses préférences individuelles. Ce système de recommandations favorise alors une découvrabilité personnalisée d'entités recommandables.

Compte tenu de l'importance de la qualité de l'information, l'algorithme de Tournesol doit être robuste et sécurisé. En particulier, l'algorithme doit être robuste à la contamination de données, aux attaques coordonnées de manipulation de l'information ou attaques bizantines. De même, la plateforme doit être sécurisé et garantir l'intégrité du processus et des données soumises par les contributeur, ainsi que respecter l'éventuel anonymat choisi des préférences individuelles.


Au delà de proposer des recommandations, Tournesol a pour ambition de constituer une base de données éthique et collaborative de grande envergure, afin de devenir un standard de facto et de montrer qu'il est possible d'allier éthique et IA et de guider les entités privés à emboîter le pas. Pour cela, un des objectifs finaux de Tournesol est de promouvoir des données collaboratives éthiques en open data et spécifiquement à destination des chercheurs en science des données.

\subsection{Modélisation statistique}

Dans le cadre d'une modélisation statistique bayésienne, sous hypothèse qu'il existe pour chaque entité un score réel avec une relation d'ordre, soit $\theta_{a}\in\Theta=\mathbf{R}$ le paramètre à estimer représentant le score de l'entité a.

Le contributeur n compare deux entités a et b en notant selon un critère principal et un ou plusieurs critères secondaires optionnelles sur une échelle de $R_{min}=-10$ à $R_{max}=10$.

Une comparaison est donc un 4-uplet (n, a, b, liste de couple notation-critère (r,c))

Le score utilisé pour la recommandation est actuellement calculé sur le critère principal et les scores sur les autres critères secondaires sont calculées de manière approximativement similaire. Ainsi, la suite de ce rapport se focalisera sur le calcul du score du critère principal à partir des notations sur ce critère principal.

\begin{figure}[ht]
  \includegraphics[width=\linewidth]{comparison.png}
  \caption{Exemple d'une comparaison entre deux vidéos}
  % \vspace{-15pt}
\end{figure}
\subsubsection{Approche bayésienne pour un problème inverse}

La plateforme Tournesol collecte les préférences déclarées des contributeurs et cherche à en déduire les préférences réelles à partir de ces observations. L'inférence des valeurs réelles à partir des observations constitue le coeur des problèmes inverses, à partir d'une transformation linéaire ou non entre observations et valeurs réelles. Pour cela, l'approche bayésienne permet d'estimer ces valeurs réelles en effectuant quelques hypothèses.

\begin{hypothesis}\label{hyp1}
Les recommandations formulées par un contributeur sont à un bruit près les véritables préférences du contributeur.
\end{hypothesis}
\begin{hypothesis}\label{hyp2}
Pour des raisons de parcimonie de calcul, le bruit est considéré gaussien et indépendant temporellement entre chaque comparaison d'un même contributeur. Cette hypothèse simplificatrice peut être remise en compte et améliorée. Dans une première version, Tournesol adopte cette hypothèse simplificatrice afin de pouvoir proposer un premier système de recommandation robuste.
\end{hypothesis}

\begin{hypothesis}\label{hyp3}

Les recommandations d'un contributeur sont supposées indépendantes. Soit $R_{ab}$ la variable aléatoire discrète dans $[\![-10,10]\!]$ représentant la note entre A et B, $(R_{ab})_{ab \in AB} \sim P_{\theta^\dagger_{ab}}$ avec AB l'ensemble des recommandations du contributeur, par le lemme des coalitions toute transformation de  $(R_{ab})_{ab \in AB}$ garde l'indépendance.

\end{hypothesis}


\subsubsection{Modélisation du problème inverse}
Soient r (ou $r_{ab}$), la notation du contributeur n entre les entités a et b

et $\Tilde{r}$, la notation normalisé entre (-1,1) avec $\Tilde{r} =r/(1+R_{max})$ et $R_{max}=10$

et la fonction continue et dérivable $\ell : (-1,1) \rightarrow \mathbf{R}$ définie par $\ell : \Tilde{r} \rightarrow -\Tilde{r}/\sqrt{1-\Tilde{r}^2}$,

alors, la fonction inverse est $\Tilde{r} : \mathbf{R} \rightarrow  (-1,1) ; \Tilde{r} : \ell \rightarrow -\ell / \sqrt{1+\ell²} $ avec $r(\ell)= \Tilde{r}(\ell)(1+R_{max})$

Soit $\theta_{ab}^\dagger=\theta_a^\dagger-\theta_b^\dagger$, avec $\theta_a^\dagger, \theta_b^\dagger$ les vrais scores des entités a et b par le contributeur n, les hypothèses  \ref{hyp1} et \ref{hyp2} se traduisent par

\begin{equation}
r_{ab}=r(\theta_{ab}^\dagger)+\zeta_{ab}    
\end{equation}


avec $\zeta_{ab}$ un bruit gaussien centré d'écart type $(1+R_{max})\sigma_0$\footnote{Le choix est fait ici pour simplifier les calculs à défaut d'\textit{a priori}, il peut être affiné par la suite}.
En appliquant $\ell$ à $\Tilde{r}(\theta_{ab}^\dagger)$, $\theta_{ab}^\dagger = \ell (\frac{( r_{ab}-\zeta_{ab})}{(1+R_{max})})$.
Cette quantité s'approche avec un développement limité au premier ordre par $\ell(\Tilde{r_{ab}}) - \ell'(\Tilde{r_{ab}})\frac{\zeta_{ab}}{(1+R_{max})}$ avec $\ell'(\Tilde{r})=(1-\Tilde{r}^2)^{-3/2}$

\begin{equation}
    \theta_{ab}^\dagger \approx \ell(\Tilde{r_{ab}}) - \ell'(\Tilde{r_{ab}})\frac{\zeta_{ab}}{(1+R_{max})}
\end{equation}

$\zeta_{ab}$ est une variable aléatoire gaussienne centrée d'écart type $(1+R_{max})\sigma_0$  par hypothèse
et $\theta_{ab}^\dagger$ et $r_{ab}$ sont des observations de variables aléatoires respectivement $\theta$ et R suivant des lois \textit{a priori} dans $\mathbf{R}$ et R à support discret.

\subsubsection{Résolution du problème inverse}

Les recommandations d'un contributeur donne des informations sur l'ensemble des $\set{\theta^\dagger_a : a \in A}$, avec A l'ensemble des entités notés.
En particulier, une recommandation entre a et b est à la fois une contribution pour $\theta^\dagger_a$ et à la fois pour $\theta^\dagger_b$.

Dans cette partie, R est le vecteur aléatoire des observations empilés.
Soit $\theta$, le vecteur aléatoire des $\theta^\dagger$ empilés
et soit $\pi(\theta)$ le prior de $\theta$ comme le produit de loi normale d'écart-type $\sigma_0$.

La formule de Bayes donne  
\begin{equation}
\pi(\theta|R) = \frac{f(R|\theta)\pi(\theta)}
{\int_{\Theta} f(R|\theta)\pi(\theta) \,d\theta }
\end{equation}

ou  

\begin{equation}
\pi(\theta|R) \propto f(R|\theta)\pi(\theta)
\end{equation}

avec

\begin{equation}
 f(R=r|\theta=\theta^\dagger)= \prod_{ab \in AB} \frac{1}{\sigma_0\ell'(\Tilde{r_{ab}})\sqrt{\tau}}\exp \left( -(\frac{\ell(\Tilde{r_{ab}})-\theta_{ab}^\dagger}{\sqrt{2}\sigma_0\ell'(\Tilde{r_{ab}})})^2 \right)    
\end{equation}
  avec  $\tau=2\pi$

\begin{equation}
 \pi(\theta)= \prod_{a \in A} \frac{1}{\sigma_0\sqrt{\tau}}\exp \left( -(\frac{\theta^\dagger_a}{\sqrt{2}\sigma_0})^2 \right)    
\end{equation}


%$  -\ln{P(\ell|\theta^\dagger)}= \sum_{(a,b)\in AB} \frac{k_{ab}}{2\sigma_0^2}  (\ell(\Tilde{r_{ab}})-\theta_{ab}^\dagger)^2 + K $ avec K une constante réelle et AB l'ensemble des comparaisons entre deux paires d'entités%
La log-vraisemblance \textit{a posteriori} s'écrit avec une constante K ne dépendant pas de $\theta$ :

\begin{equation}
 log \mathcal{L} \triangleq \log{\pi(\theta|R)} = log(f(R|\theta)) + log(\pi(\theta))+ K     
\end{equation}


\begin{equation}
 log \mathcal{L} = log(f(R|\theta)) + log(\pi(\theta))+ K     
\end{equation}

\begin{equation}
 log \mathcal{L} = \frac{1}{2\sigma_0^2} \left( \sum_{ab \in AB} k_{ab}(\ell(\Tilde{r_{ab}})-\theta_{ab}^\dagger)^2 + \sum_{a \in A}\theta^\dagger_a^2 \right) + K'     
\end{equation}

avec $k_{ab} = \frac{1}{\ell'(\Tilde{r_{ab}})^2}$,

L'estimateur du maximum \textit{a posteriori} se déduit des conditions du premier ordre de la log-vraisemblance \textit{a posteriori} car le problème est fortement convexe.

\begin{equation}
\forall a \in A, \sigma_0^2\frac{\partial{log\mathcal{L}}}{\partial{\theta_a^\dagger}} = 0 = \theta^\dagger_a + \sum_{ab \in AB}
\end{equation}
\subsection{Fonctionnement de l'algorithme Mehestan}

L'algorithme Mehestan fonctionne en trois phases :

\begin{enumerate}
    \item Construction des scores individuels
    \item Construction du réétalonnage et des scores normalisés
    \item Construction des scores des entités
\end{enumerate}
\subsubsection{Construction des scores individuels}

Pour chaque contributeur, à partir de sa matrice de comparaison $r_t$, la première partie de l'algorithme calcule le vecteur des scores individuels bruts et des incertitudes à partir des formules énoncées précédemment

\subsubsection{Construction du réétalonnage et des scores normalisés}

Avant d'agréger les scores individuels, il est nécessaire de réétalonner ces scores sur une même échelle. L'algorithme calcule un noyau d'utilisateur de référence puis calcule tous les réétalonnages de chaque utilisateur par rapport à ce noyau. Pour cette étape, l'algorithme utilise \gls{qrmed}

\subsubsection{Construction des scores des entités}

L'algorithme calcule les scores globaux en utilisant \gls{qrmed} sur les scores normalisés des contributeurs pour chaque entité.


\pagebreak


\section{Heuristiques en ligne}

\pagebreak
\section{Conclusion}

La démocratisation de l'\gls{ia} produisant de plus en plus de résultats spectaculaires comme DALL-E2, Midjourney et Stable Diffusion, s'accélère à grand pas ces derniers mois, à tel point que se pose la question du futur de la composition artistique. Plus les avancées technologiques progressent, plus il apparaît important de prendre le temps de réfléchir à leur sens, à leur utilisation pour le service du bien commun. Ce temps de pause doit permettre d'évaluer, d'améliorer l'existant pour en faire des solutions plus éthiques. Pour cela, il est nécessaire d'abonder le champ de la recherche dans l'éthique de l'\gls{ia} et les projets mettant en oeuvre une démarche éthique.

Dans cette optique, comprendre puis enrichir le projet de l'association Tournesol constitue une premier pas vers la recherche dans l'éthique de l'\gls{ia}. En effet, ces travaux ont été l'occasion de réfléchir sur l'éthique, d'en saisir les tenants et les aboutissants. Cela passe par la compréhension de la modélisation statistique et de l'algorithme Mehestan, puis par la mise en place d'une heuristique en ligne. Cette nouvelle fonctionalité s'insère dans un existant en production, tout en se synchronisant sur cet existant évoluant de jour en jour. 
Afin de la mettre en production la nouvelle fonctionnalité, il a été nécessaire de s'assurer des garanties théoriques et d'évaluer de manière pratique les écarts et approximations d'une telle heuristique.

Pour cela, dans un premier temps, des tests déterministes donnent une première estimation de la qualité de l'heuristique, à travers le biais de somme nulle et l'écart à l'algorithme Mehestan. Or, il n'est pas possible de simuler tous les cas. Ainsi, une approche par simulation de Monte-Carlo donnent une image plus précise des écarts. Enfin, une dernière approche a été mise en place en jouant aléatoirement un tir sur données réelles et à volumétrie importante.

Les résultats sont satisfaisants par rapport aux enjeux, bien que le temps de calcul de l'heuristique ne permet pas une intégration sur tous les contributeurs, notamment au fur et à mesure que la base de données devient de plus en plus conséquent. Après audit du code, l'équipe de Tournesol songe à activer l'heuristique pour les contributeurs ayant le plus de comparaisons, afin de permettre de mettre à jour en direct les scores de Tournesol et d'afficher une actualisation en temps réel plutôt que toutes les six heures.
\pagebreak








\pagenumbering{Roman}

\phantomsection
\addcontentsline{toc}{section}{Bibliographie}
\section*{Bibliographie}

\nocite{*}

\printbibliography[
    heading = subbibintoc,
    filter=papers,
    title={Rapports institutionnels et articles scientifiques}]
    
\printbibliography[
    heading = subbibintoc,
    filter=onlinepapers,
    title={Ressources en ligne}]

\printbibliography[
    heading = subbibintoc,
    type=book,
    title={Livres}]
    
\pagebreak




%%%%%%%%%%%%%%%%%%%%%%%%%%%%%%%%%%%%%%%%%
%%%%%%%%%%%%%%%%%%%%%%%%%%%%%%%%%%%%%%%%%
%%%%%%%%%%%%%%%%%%%%%%%%%%%%%%%%%%%%%%%%%
%%%%%%%% APPENDICES %%%%%%%%%%%%%%%%%%%%%
%%%%%%%%%%%%%%%%%%%%%%%%%%%%%%%%%%%%%%%%%
%%%%%%%%%%%%%%%%%%%%%%%%%%%%%%%%%%%%%%%%%
%%%%%%%%%%%%%%%%%%%%%%%%%%%%%%%%%%%%%%%%%


\begin{appendices}
\section{Etat des lieux des définitions sur l'intelligence artificielle}\label{appendice:definition-ia}

\begin{tcolorbox}[title= définitions anglaises du Cambridge dictionary]
\begin{enumerate}
    \item {\textbf{intelligence} \\
\textit{noun} \\
the ability to learn, understand, and make judgments or have opinions that are based on reason}
    \item {
    \textbf{artificial} \\
\textit{adjective}\\
made by people, often as a copy of something natural
    }
    \item  {
    \textbf{artificial intelligence}\\
    \textit{noun}\\
the study of how to produce machines that have some of the qualities that the human mind has, such as the ability to understand language, recognize pictures, solve problems, and learn.
    }
\end{enumerate}





\end{tcolorbox}


\begin{tcolorbox}[title= Définitions françaises du dictionnaire ATILF]
\begin{enumerate}
    \item {\textbf{INTELLIGENCE}, \textit{subst. fém.}\\
I. [Chez les êtres animés] Fonction mentale d'organisation du réel en pensées chez l'être humain, en actes chez l'être humain et l'animal.
}
    \item {\textbf{ARTIFICIEL, ELLE}, \textit{adj. et subst.} \\
I. \textit{Adjectif} \\
A. Qui est dû à l'art, qui est fabriqué, fait de toutes pièces; qui imite la nature, qui se substitue à elle; qui n'est pas naturel.
}
    \item {
\textit{LOG., INFORM.} \textbf{Intelligence artificielle}, Recherche de moyens susceptibles de doter les systèmes informatiques de capacités intellectuelles comparables à celles des êtres humains`` (La Recherche, janv. 1979, no 96, vol. 10, p. 61).
}
\end{enumerate}



\end{tcolorbox}


\begin{tcolorbox}[title= Définitions françaises des dictionnaire Littré]
\begin{enumerate}
    \item {\textbf{intelligence}
\textit{(in-tèl-li-jan-s') s. f.}
Qualité de ce qui est intelligent ; faculté de comprendre.
}
    \item {\textbf{artificiel, elle}
\textit{(ar-ti-fi-si-èl, è-l') adj.}
Qui se fait par art, opposé à naturel.
}
\end{enumerate}

\end{tcolorbox}
\pagebreak

\section{Etat des lieux des définitions sur l'éthique des \gls{ia}}

\begin{tcolorbox}[title= Définitions françaises de l'éthique du dictionnaire ATILF et Littré]
\begin{enumerate}
\item{
\textbf{ÉTHIQUE}, \textit{subst. fém. et adj.} \\
I. Subst. fém.\\
A. PHILOS. Science qui traite des principes régulateurs de l'action et de la conduite morale
}
\item{
\textbf{éthique} \\
\textit{(é-ti-k') s. f.} \\
1 Terme de philosophie. La science de la morale.
L'éthique politique a deux objets principaux : la culture de la nature intelligente, l'institution du peuple, Diderot, Opin. des anc. philos. (Sarrasins).
Les Éthiques, titre d'un ouvrage d'Aristote qui traite de la morale.
Tel est le traité des caractères de mœurs que nous a laissé Théophraste ; il l'a puisé dans les Éthiques d'Aristote, dont il fut le disciple, La Bruyère, Disc. sur Théophr.

2 Adj. Qui appartient à la morale. Préceptes éthiques.
}
\end{enumerate}
\end{tcolorbox}



\pagebreak

\section{Variante de l'heuristique dans le cas 3x3}\label{appendice:calcul-theorique}

Le contributeur effectue une nouvelle comparaison entre $E_{1}$ et $E_{3}$ avec une notation $n_{1-3} \in [\![-10;10]\!]$. Les scores sont cette fois-ci calculés par l'heuristique.

\begin{align*}
r_{t+1}= \begin{pmatrix}
0 & n_{1-2} &  n_{1-3} \\
-n_{1-2} & 0 & 0\\
 -n_{1-3}  & 0 & 0
\end{pmatrix}
~~~~~~
l_{t+1}= \begin{pmatrix}
0 & l_1 & l_2\\
-l_1 & 0 & 0\\
-l_2 & 0 & 0
\end{pmatrix}
~~~~~~
k_{t+1}= \begin{pmatrix}
0 & k_1 &  k_2 \\
k_1 & 0 & 0 \\
 k_2 & 0 & 0
\end{pmatrix}
\end{align*}

\begin{align*}
l_{t+1} \cdot k_{t+1}= \begin{pmatrix}
0 & l_1 k_1 & l_2 k_2\\
-l_1 k_1 & 0 & 0\\
-l_2 k_2 & 0 & 0
\end{pmatrix}
\end{align*}

\begin{align*}
L_{t+1}= \begin{pmatrix}
l_1 k_1 +l_2 k_2 \\
-l_1 k_1\\
-l_2 k_2
\end{pmatrix} \triangleq
\begin{pmatrix}
L1\\
L2\\
L3
\end{pmatrix} 
~~~~~~
K_{aa,t+1}= \begin{pmatrix}
\alpha + k_1 + k_2\\
\alpha +  k_1 \\
\alpha + k_2
\end{pmatrix} \triangleq
\begin{pmatrix}
K1\\
K2\\
K3
\end{pmatrix} 
\end{align*}




Pour $\tau=0, A^{t+1,0}=\set{E_1,E_3}$

\begin{align*}
  \scoreh_{t+1,0}= \begin{pmatrix}
(L_1+\score_{2} k_1) / K_1 \\
(L_3+\score_{1} k_2) / K_3
\end{pmatrix}  
\end{align*}

avec $\score_{1} = - \score_{2} = l_1 k_1/(\alpha + 2 k_1)$

Pour $\tau=1, A^{t+1,1}=\set{E_1,E_2,E_3}$

\begin{align*}
\scoreh_{t+1,1}= \begin{pmatrix}
(L_1+\score_{2} k_1 + \scoreh_{t+1,0,3} k_2) / K_1 \\
(L_2+\scoreh_{t+1,0,1} k_1) / K_2 \\
(L_3+\scoreh_{t+1,0,1} k_2) / K_3 
\end{pmatrix}    
\end{align*} 


\begin{align*}
 BSN =
(L_1+\score_{2} k_1 + \scoreh_{t+1,0,3} k_2) / K_1 +
(L_2+\scoreh_{t+1,0,1} k_1) / K_2 +
(L_3+\scoreh_{t+1,0,1} k_2) / K_3 
\end{align*}

\begin{equation*}
    \forall{n_{1-2},n_{1-3}}\in [\![-10;10]\!]² , BSN \in [-1.10,1.10]
\end{equation*}

\begin{figure}[ht]
  \includegraphics[width=\linewidth]{heatmap2.png}
  \caption{Distribution des résultats de l'indicateur BSN (ordonnée,abscisse)}
  % \vspace{-15pt}
\end{figure}

Le minimum et le maximum sont atteints respectivement en $(n_{1-2},n_{1-3})=(4, -9)$ et en (-4, 9).


Les scores de l'algorithme Mehestan sont données par les formules suivantes :

$K_{t+1} \score_{t+1}= L_{t+1}$

$\score_{t+1}=K^{-1}_{t+1} L$

avec 
$ K_{t+1}= \begin{pmatrix}
K_1&-k_{12}&-k_{13}\\
-k_{12}&K_2&-k_{23}\\
-k_{13}&-k_{23}&K_3
\end{pmatrix}$

ici, $k_{12}=k_1, k_{13}=k_2, k_{23}=0$
alors, 

$ K_{t+1}= \begin{pmatrix}
K_1&-k_1&-k_2\\
-k_1&K_2&0\\
-k_2&0&K_3
\end{pmatrix}$

$ K^{-1}_{t+1}= \left( \begin{array}{ccc} K_1 K_2 & k_1 K_2 & k_2 K_1 \\ k_1 K_2 & K_2 K_3-k_2^2 & k_1 k_2 \\ k_2 K_1 & k_1 k_2 & K_1 K_3-k_1^2 \end{array} \right)/(-K_2 k_1^2-k_2^2 K_1+K_1 K_2 K_3)$

EAM=  $|\score_{t+1,1} - \scoreh_{t+1,1,1}|   +  |\score_{t+1,2} - \scoreh_{t+1,1,2}|  +  |\score_{t+1,3} - \scoreh_{t+1,1,3}|  $



\begin{equation*}
\forall{n_{1-2},n_{2-3}}\in [\![-10;10]\!]²,  EAM \in [0.000,1.640 ]
\end{equation*}


Le minimum et le maximum sont atteints respectivement en $(n_{1-2},n_{1-3})=(0, 0)$ et en (-9, 7).

\section{Résultats de l'évaluation pratique de l'heuristique}

\subsection{Profil uniforme}

\begin{figure}[h!]
    \begin{subfigure}{\textwidth}
        \includegraphics[scale=0.8] {score_20_10_uniform_220926_meaplus_2}
    \end{subfigure}
    \bigskip
    \begin{subfigure}{\textwidth}
        \includegraphics[scale=0.8] {score_20_10_uniform_220926_meaplus_3}
    \end{subfigure}
    \caption{Résultats de 10 ajouts parmi 20 entités avec un prior uniforme discret }
\end{figure}

\begin{figure}[h!]
    \begin{subfigure}{\textwidth}
        \includegraphics[scale=0.8] {score_20_10_uniform_220926_meaplus_4}
    \end{subfigure}
    \bigskip
    \begin{subfigure}{\textwidth}
        \includegraphics[scale=0.8] {score_20_10_uniform_220926_meaplus_5}
    \end{subfigure}
    \caption{Résultats de 10 ajouts parmi 20 entités avec un prior uniforme discret }
\end{figure}

\pagebreak

\subsection{Profil extrême}


\begin{figure}[h!]
    \begin{subfigure}{\textwidth}
        \includegraphics[scale=0.8] {score_20_10_invgaussian_220925_meaplus_2}
    \end{subfigure}
    \bigskip
    \begin{subfigure}{\textwidth}
        \includegraphics[scale=0.8] {score_20_10_invgaussian_220925_meaplus_3}
    \end{subfigure}
    \caption{Résultats de 10 ajouts parmi 20 entités avec un prior extrême discret }
\end{figure}

\begin{figure}[h!]
    \begin{subfigure}{\textwidth}
        \includegraphics[scale=0.8] {score_20_10_invgaussian_220925_meaplus_4}
    \end{subfigure}
    \bigskip
    \begin{subfigure}{\textwidth}
        \includegraphics[scale=0.8] {score_20_10_invgaussian_220925_meaplus_5}
    \end{subfigure}
    \caption{Résultats de 10 ajouts parmi 20 entités avec un prior extrême discret }
\end{figure}


\pagebreak

\subsection{Profil gaussien discret}

\begin{figure}[h!]
    \begin{subfigure}{\textwidth}
        \includegraphics[scale=0.8] {score_20_10_gaussian_220926_meaplus_2}
    \end{subfigure}
    \bigskip
    \begin{subfigure}{\textwidth}
        \includegraphics[scale=0.8] {score_20_10_gaussian_220926_meaplus_3}
    \end{subfigure}
    \caption{Résultats de 10 ajouts parmi 20 entités avec un prior gaussien discret }
\end{figure}


\begin{figure}[h!]
    \begin{subfigure}{\textwidth}
        \includegraphics[scale=0.8] {score_20_10_gaussian_220926_meaplus_4}
    \end{subfigure}
    \bigskip
    \begin{subfigure}{\textwidth}
        \includegraphics[scale=0.8] {score_20_10_gaussian_220926_meaplus_5}
    \end{subfigure}
    \caption{Résultats de 10 ajouts parmi 20 entités avec un prior gaussien discret }
\end{figure}




\section{Résultats des tirs sur données publiques}

\end{appendices}
\end{document}