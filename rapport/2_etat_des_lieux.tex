\section{Etat des lieux de l'\gls{ia} et de l'éthique}

Cette section introduit les définitions qui seront utilisées dans ce rapport et qui sont indispensables à la compréhension du sujet. Comme en sociologie, les définitions sont un préalable pour une compréhension commune et éclairée et permettent la délimitation du périmètre des sujets. Au delà de l'aspect mathématique et statistique des algorithmes présentés dans ce rapport, il est tout aussi important de se questionner sur l'éthique des algorithmes et modèles déjà existants. \textit{A priori}, les termes utilisés seront, sauf exception, en français, bien que les équivalents anglophones soient davantage connus et usités. Le glossaire, en début de rapport, liste les équivalences.

\subsection{L'\gls{ia} : un, deux, trois, soleil ?}

\subsubsection{La génèse de l'\gls{ia}}
Aujourd'hui, l'\gls{ia} est un terme célèbre et hautement médiatisé, bien connu des citoyens sans toutefois avoir une idée claire de toutes les réalités que le terme désigne. Pourtant composé de deux termes \textit{a priori} anodins, l'\gls{ia} désigne tour à tour pour l'imaginaire collectif des robots, des machines, une réalité fantasmée ou crainte où ces entités sont soit au service de l'humanité ou soit source d'aliénation de cette dernière, à l'image des différents scénarios de films de science-fiction.


Né dans la première moitié du \siecle{20}, avec l'essor de la cybernétique, l'\gls{ia} désigne en premier lieu une discipline scientifique ayant pour finalité de créer des machines capables d'effectuer des tâches complexes, automatisées et guidées par des décisions éclairées en fonction des informations disponibles, à l'image de l'intelligence humaine.

Plus récemment, avec les progrès de l'informatique, de la collecte et du stockage des données, et des avancées scientifiques dans le domaine de la statistique, le grand public a uniquement connaissance des intelligences artificielles les plus médiatisées, comme AlphaGo et DeepBlue. Beaucoup d'\gls{ia} sont complètement inconnues, prenant des décisions mondialement importantes, telles qu'Aladdin de Blackrock pour la gestion automatisée d'actifs financiers à grande échelle, ou des décisions à haut risque telles que Kargu-2 l'\gls{ia} des drones autonomes tueurs. Le changement d'échelle entre le premier algorithme informatique et ces dernières \gls{ia} amène naturellement à se poser la question de l'éthique des algorithmes et des \gls{ia}, discutée dans la sous-section suivante \ref{subsection:ethique}.

Tout d'abord, le terme "intelligence" dérive du latin \textit{intelligencia}, référant à la compréhension et à la connaissance. L'intelligence réfère en général à l'intelligence humaine, i.e. cognitive, comme la capacité à formuler des pensées pour raisonner, anticiper ou s'adapter, faire, se souvenir ou créer.
Enfin, le terme "artificielle" est dérivé du latin \textit{artificialis}. Ce terme est composé du terme art du latin \textit{ars}, désignant l'ensemble des connaissances et des techniques pour atteindre un objectif, et du terme latin \textit{facio}, traduisant l'action de faire. À la fois le terme renvoie à la façon de faire dans l'état de l'art dans le sens premier et à la fois plus tardivement à la façon de contrefaire la nature.

En somme, l'\gls{ia} est l'ensemble des méthodes et des connaissances pour effectuer à l'aide d'entité artificielle des activités humaines simples ou complexes, impliquant une activité cognitive pouvant aller de la perception d'information jusqu'à la prise de décision par un raisonnement éclairé. En ce sens, un algorithme prend en entrée des informations, procède à l'ensemble de ses instructions, souvent transposées d'activités cognitives humaines comme le raisonnement déterministe ou bayésien et rend compte d'un ou plusieurs résultats.

\subsubsection{Une définition opératoire arbitraire}
Afin de délimiter un premier périmètre, ce présent rapport donnera la définition suivante de l'\gls{ia} et se penchera sur l'entité opérationnelle plutôt que la discipline scientifique \footnote{un ensemble de définitions est disponible en annexe \ref{appendice:definition-ia}} : une \gls{ia} est une entité informatisée ayant des fonctions relevant du traitement de l'infomation servant ou pouvant servir à la prise de décision. Tous les algorithmes informatisés sont inclus dans cette définition, y compris un simple "hello world". Dans ce dernier cas, l'\gls{ia} est rudimentaire car elle retourne presque surement l'information "hello world", ou n'importe quelle même résultat de manière systèmatique, quoi que soit ses entrées. Même si l'information est de nature simple, la présence de ce résultat permet de déduire du bon fonctionnement de l'algorithme et peut être source de décision si par exemple le résultat sert pour s'assurer du bon fonctionnement ("healthcheck").

D'autre part, la classe des algorithmes informatisés comprend l'ensemble des méthodes d'apprentissage statistique (ou \gls{ml} en anglais) dont les méthodes utilisant des réseaux de neurone profond (ou \gls{dl} en anglais). Aujourd'hui, aux yeux du public, ces méthodes, en particulier le \gls{dl}, sont confondues avec l'\gls{ia} car leurs performances époustouflantes sont davantage sujettes à être exposées et promues par les médias et à les étiqueter comme l'\gls{ia} à part entière. 

* autoapprenant ?

D'autres exemples d'IA :
* IA dans les voitures semi-autonomes
* 1.2.3 soleil
* système de recommandations
* matching

* IA_faible/forte




\subsection{L'éthique : des graines à semer}\label{subsection:ethique}

Alors que la notion d'\gls{ia} posait des problèmes en termes de représentation unifiée par chacun, l'éthique est une notion encore plus difficile à définir. En effet, l'éthique renvoie à la fois à plusieurs réalités, et également
\cite{ethique-cnil}
\pagebreak

* maximiser la rétention des utilisateurs => IA exploite les failles psychologique => l'addiction et les vidéos addictives, sujets viraux qui exploite la colère et la haine // système de récompense

* neutralité ?
* déontologique vs conséquentialiste