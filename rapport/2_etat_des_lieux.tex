\section{Etat des lieux de l'\gls{ia} et de l'éthique}

Cette section introduit les définitions qui seront utilisées dans ce rapport et qui sont indispensables à la compréhension du sujet. Comme en sociologie, les définitions sont un préalable pour une compréhension commune et éclairée et permettent la délimitation du périmètre des sujets. Au delà de l'aspect mathématique et statistique des algorithmes présentés dans ce rapport, il est tout aussi important de se questionner sur l'éthique des algorithmes et modèles déjà existants. \textit{A priori}, les termes utilisés seront, sauf exception, en français, bien que les équivalents anglophones soient davantage connus et usités. Le glossaire, en début de rapport, liste les équivalences.

\subsection{L'\gls{ia} : un, deux, trois, soleil ?}

Aujourd'hui, l'\gls{ia} est un terme célèbre, bien connu des citoyens sans toutefois avoir une idée claire des réalités que le terme désigne. Né dans la première moitié du \siecle{20}, avec l'essor de la cybernétique, l'\gls{ia} désigne let par extension sa discipline scientifique

\subsection{L'éthique : des graines à semer}

Alors que la notion d'\gls{ia} posait des problèmes en termes de représentation unifiée par chacun, l'éthique est une notion encore plus difficile à définir. En effet, l'éthique renvoie à la fois à plusieurs réalités, et également
\cite{ethique-cnil}
\pagebreak