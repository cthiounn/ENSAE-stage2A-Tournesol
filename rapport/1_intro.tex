\section{Introduction}

Inondée par un flux continu d'informations diverses et potentiellement contradictoires, qu'elles proviennent d'un clair ruisseau, d'un fleuve pollué ou d'un torrent diluvien, l'humanité doit aujourd'hui résoudre l'un des plus grands défis collectifs qui lui a été posé, à savoir l'émergence d'un ensemble de crises écologiques, sanitaires et sociales, toutes aussi interconnectées que dangereuses. Malédiction ou paire d'atout, l'information découverte par la science et les sociétés à travers les siècles, par les scientifiques et les citoyens, doit permettre grâce à ce bien commun de faire émerger la meilleure solution pour répondre efficacement à ces crises. 

Encore plus aujourd'hui, l'information utilisée est source de toutes nos décisions. L'information, comme preuve intangible ou comme élément rationnel de croyance, peut aider les citoyens à mieux forger leur conception du monde et de la vérité avec une logique bayésienne. Elle est donc centrale dans le quotidien des citoyens, dans la prise de décision manuelle et individuelle, ainsi que dans la prise de décision collective et automatisée. L'automatisation de ces décisions collectives a par corollaire un impact important sur ses citoyens sujets et doit tendre à être le plus "juste" possible.

Néanmoins, la "justesse" des décisions prises par un algorithme s'apprécie à l'aune de la définition que chacun pourrait lui donner et peut donc se caractériser de diverses manières, notamment d'un point de vue utilitariste ou d'un point de vue de l'éthique. D'un côté, l'utilitarisme permet de maximiser l'utilité agrégée sans toutefois prendre en compte l'équité. De l'autre côté, l'approche "éthique", \textit{a contrario}, considère plusieurs critères moraux et va au delà de l'égalité de droit en proposant une réflexion normative en avance sur le droit. De manière pratique, une décision éthique, comme pour l'éthique professionnelle, tend à rester fidèle à plusieurs valeurs (fidélité à ses sujets, fiabilité, transparence). En ce sens, dans un système démocratique, une décision éthique doit rendre compte du vote démocratique si exprimé, sinon du respect des valeurs démocratiques (liberté, égalité, fraternité, équité, solidarité).

Or, aujourd'hui, tous les algorithmes apprenants et de manière plus générale, l'\Gls{ia}, à notre modeste connaissance, sont entraînés avec des données comportant des biais statistiques (de selection, de mesure) ou non statistiques (racisme, sexisme, discrimination). Ceci amène les modèles à reproduire ces biais et à formuler des recommandations et des décisions non adaptées, aux conséquences potentiellement dangereuses. De surcroît, les données sont collectées de manière non contrôlée (scrapping en ligne de textes de communautés particulières, données dangereuses non vérifiées) et peuvent faire l'objet de compromission par diverses entités, dans le but de manipuler un résultat ou de se dégager d'une quelconque responsabilité.

Afin d'améliorer l'éthique des \gls{ia}, l'association Tournesol a construit une plateforme collaborative \href{https://tournesol.app}{https://tournesol.app}, où les internautes sont invités à comparer selon différents critères plusieurs choix entre eux (vidéos, candidats), afin de mettre en avant l'entité qui leur semble la plus recommandable. Les données de cette plateforme permettent de mettre en place un système de recommandation robuste fondé sur un score calculé de manière transparente, robuste et sécurisé.

Cependant, les comparaisons parmi un nombre important de choix n'engendrent que des données parcimonieuses. Dès lors, l'enjeu est de reconstruire au niveau agrégé et de manière robuste tous les scores relatifs puis d'élaborer des scores absolus pour pouvoir mettre en avant l'information jugée primordiale. En somme, l'enjeu est de pouvoir quantifier la désirabilité relatif des entités à partir d'une poignée de préférences révélées de manière robuste et sécurisé.

Pour cela, l'algorithme Mehestan sur les données de Tournesol tire parti d'estimateurs basées autour de la \gls{qrmed}. Néanmoins, ces calculs ne sont pas assez rapides pour proposer des résultats au contributeur en temps réel. Afin de préserver l'engagement du contributeur, il est souhaitable d'afficher en temps réel sa participation et sa contribution au score calculé en temps réel à l'aide d'heuristiques en ligne tout en garantissant la sécurité et la fiabilité de ces calculs.

\pagebreak